\documentclass{article}

\title{Decentralised verfication of real world events}
\date{\today}
\author{remans, Verethy}

\begin{document}
	\maketitle
	
	\begin{abstract}
		The underlying mechanics and mathematical framework for a decentralised concensus protocol on real world events are discussed.
	\end{abstract}
	
	\section{Introduction}
		Blockchain technology has flourished in recent years and enabled the widespread creation of new cryptocurrencies. The idea of trustless money transfer was then sought to be extended. With the invention of smart contracts, the decentralised consensus idea was no longer solely focused on currencies and allowed the realisation of almost any decentralised, democratic process imaginable. 
		
		However, all of these applications are focused on the blockchain or decentralised system they run on and it is to date not possible to connect the trustless system to the real world. Exchange rates to fiat currencies and other real world based information cannot be implemented into the trustless world, since they all rely on some centralised source, a provider, of information. Even though exchange rates, for example, undergo some sort of consensus on the markets, the rate shown on a website or in a newspaper is not necessarily the same as everywhere else. In the worst case a malicious provider might show fraudulent information for their own benfit.
		
		In order to tackle this problem, this paper proposes a decentralised and trustless consensus protocoll to decide on the "state" of the real world, or parts of it, at a given time. If this consensus can be reached on the blockchain, the state of the real world, as seen from a trustless point of view, can be trustlessly incorporated into the chain and therefore be considered as valid as any other information stemming from the chain itself.
		
	 
\end{document}